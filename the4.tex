\documentclass[11pt]{article}
\usepackage[utf8]{inputenc}
\usepackage{float}
\usepackage{amsmath}
\usepackage{amssymb}

\usepackage[hmargin=3cm,vmargin=6.0cm]{geometry}
%\topmargin=0cm
\topmargin=-2cm
\addtolength{\textheight}{6.5cm}
\addtolength{\textwidth}{2.0cm}
%\setlength{\leftmargin}{-5cm}
\setlength{\oddsidemargin}{0.0cm}
\setlength{\evensidemargin}{0.0cm}

% symbol commands for the curious
\newcommand{\setZp}{\mathbb{Z}^+}
\newcommand{\setR}{\mathbb{R}}
\newcommand{\calT}{\mathcal{T}}

\begin{document}

\section*{Student Information } 
%Write your full name and id number between the colon and newline
%Put one empty space character after colon and before newline
Full Name : Berkin Kerim KONAR \\
Id Number : 2375343 \\

% Write your answers below the section tags
\section*{Answer 1}

Answer:\\

    We first choose 1 star 8 unhabitable planets and, 2 habitable planets.\\
    
    $\binom{10}{1} * \binom{80}{8} * \binom{20}{2}$\\
    
    Now  when it comes to place the planets in way such that at least 6 unhabitable planets are between 2 habitable planets we compute the total combinations in 3 parts. (Since the star is in the middle we don't need to find a place for it.)\\
    
    Part 1: suppose there are 6 unhabitable. planets between the habitable ones.\\
    
    We choose 6 unhabitable in between and order them.\\
    
    $\binom{8}{6}*6!*2!*3*2!$. (There 3 ways to distribute 2 planets to 2 sides of the inner planets)\\
    
    Part 2: suppose there are 7 unhabitable. planets between the habitable ones.\\
    
    We choose 7 unhabitable in between and order them.\\
    
    $\binom{8}{7}*7!*2!*2*1!$. (There are 2 ways to distribute 1 planet to 2 sides of the inner planets)\\
    
    Part 3: suppose there are 8 unhabitable. planets between the habitable ones.
    
    We choose 7 unhabitable in between and order them.\\
    
    $\binom{8}{8}*8!*2!*0!$. (There is 1 way to distribute 0 planet to 2 sides of the inner planets)\\
    
    So in total we have $\binom{8}{6}*6!*2!*2! + \binom{8}{7}*7!*2!*1! + \binom{8}{8}*8!*2!*0!$ oderings of planets. When we multiply this result with the amount of choices we get answer.\\
    
    $\binom{10}{1} * \binom{80}{8} * \binom{20}{2} * (\binom{8}{6}*6!*2!*3*2! + \binom{8}{7}*7!*2!*2*1! + \binom{8}{8}*8!*2!*0!) = 2.6648127*10^{19}$\\
    
\section*{Answer 2}

Answer:\\
    
    We get the solution to the homogeneous recurrence relation $a_n = 2a_{n-1} + 15an_{n-2} - 36a_{n-3}$ by assuming $a_n = r^n$ and solving the corresponding homogeneous equation $r^3 = 2r^2 + 15r - 36$.\\
    
    $r^3 - 3r^2 + r^2 -15r +36 = r^2(r-3) + (r-3)*(r-12) = (r-3)*(r^2+r-12) = (r-3)^2*(r+4)$\\
    
    $r_{1,2} = 3, r_3 = -4$\\
    
    So the solution to the homogeneous system is $a_n = c_13^n + c_2n3^n + c_3(-4)^n$
    
    To get the particular solution we try $c2^n$ first in the non-homogeneous recurrence relation.\\
    
    $c2^n = 2c2^{n-1} + 15c2^{n-2} -36c2^{n-3} + 2^n \rightarrow 6c2^{n-3} = 2^n$. So if we let $c = 8/6$ our guess satisfies the particular solution.\\
    
    So the solution $a_n = a^h + a^p$ is $a_n = c_13^n + c_2n3^n + c_3(-4)^n + \frac{8}{6}2^n$

\section*{Answer 3}

Answer:\\

    Let $a_n$ denote the number of valid codes of length n. Then we can divide $a_n$ to 2 parts where.\\
    
    Part 1: The last digit of the code is even.\\
    
    Then we see that the rest of the length of n-1 code contains odd number of odd numbers. Therefore it is valid (So it is equal to the $a_{n-1}$) and, there are 5 even numbers we can choose for the last digit.\\
    
    So we have $5*a_{n-1}$ combinations of length n codes that end with even digits.
    
    Part 2: The last digit of the code is odd.\\
    
    Then we see that the rest of the code length of n-1 contains even number of digits. Therefore it is invalid (So we can calculate it by subtracting the valid combinations from all of the combination which is $10^{n-1}$) and, there are 5 odd numbers we can choose for the last digit of the code length of n.\\
    
    So we have $5*(10^{n-1} - a_{n-1})$ combinations of length of n codes that end with odd digits.\\
    
    When we add these 2 parts we get solution as $a_n = 5*a_{n-1} + 5*(10^{n-1} - a_{n-1}) = 5*10^{n-1}$ except the case of $a_0 = 0$. (the number of odd numbers will always be zero therefore even in $n = 0$ case.)\\
    
    Also when we look at $a_n = 5*10^{n-1}, a_{n-1} = 5*10^{n-2}$ we see that $a_n = 10a_{n-1}$
    for $1 \leq n$ and, $a_0 = 0, a_1 = 5$.\\
    
\section*{Answer 4}
Answer:\\

    Let $G(x) = \sum_{k=0}^{\infty} a_kx^k = \sum_{k=3}^{\infty} a_kx^k + a_0 + a_1x + a_2x^2$ then we get.\\
    
    $-3xG(x) = \sum_{k=1}^{\infty} -3a_{k-1}x^k = \sum_{k=3}^{\infty} -3a_{k-1}x^k + -3a_0x + -3a_1x^2$\\
    
    $3x^2G(x) = \sum_{k=2}^{\infty} 3a_{k-2}x^k = \sum_{k=3}^{\infty} 3a_{k-2}x^k + 3a_0x^2$\\
    
    $-x^3G(x) = \sum_{k=3}^{\infty} -a_{k-3}x^k$\\
    
    after we sum all 4 equations and take $a_0 = 1, a_1 = 3, a_2 = 6$ we get\\
    
    $G(x)(1-3x+3x^2-x^3) = \sum_{k=3}^{\infty}(a_k - 3a_{k-1} + 3a_{k-2} - a_{k-3})x^k + 1$ and since $a_k - 3a_{k-1} + 3a_{k-2} - a_{k-3} = 0$ we get.\\
    
    $G(x)(1-3x+3x^2-x^3) = 1 \rightarrow G(x) = \frac{1}{(1-x)^3}$\\ 
    
    By the series formula $\frac{1}{(1-x)^n} = \sum_{k=0}^{\infty} C(n+k-1,k)x^k$ we get.\\
    
    $\frac{1}{(1-x)^3} = \sum_{k=0}^{\infty} C(3+k-1,k)x^k$\\
    
    $G(x) = \sum_{k=0}^{\infty} a_kx^k = \sum_{k=0}^{\infty} C(2+k,k)x^k$\\
    
    Therefore the solution is $a_k = C(2+k,k)$
    
    
\section*{Answer 5}
\paragraph{a.}
Answer: \\

    

    Let $((a,b),(c,d)) \in R$ then $a+d = b+c \rightarrow b+c = a+d$ then we see that.\\
    
    $((c,d),(a,b)) \in R$ therefore R is symetric .\\
    
    And if we let $c=a, d=b$ we see that $a+d = b+c \rightarrow a+b = a+b \rightarrow 0 = 0$ therefore.\\
    
    $((a,b),(a,b)) \in R$ therefore $R$ is reflexive.\\
    
    Now let $((a,b),(c,d)) \in R$ and $((c,d),(f,k)) \in R$ then we get.\\
    
    eq 1. $a+d = b+c$.\\
    
    eq 2. $c+k = d+f$.\\
    
    From these two equations we get.\\
    
    $a+k = b+f$. which implies.\\
    
    $((a,b),(f,k)) \in R$. Therefore R is transitive.
    
    Since R is reflexive, symetric and, transitive it is an equivalance relation.\\

\paragraph{b.}
Answer: \\

    The equivalance class of (1,2) with respect to R is.\\
    
    $[(1,2)]_R = \{(c,d) | ((1,2),(c,d)) \in R\}$ which implies.\\
    
    $[(1,2)]_R = \{(c,d) | 1 + d = 2 + c\}$ which implies.\\
    
    $[(1,2)]_R = \{(c,d) | d = 1 + c\}$

\end{document}