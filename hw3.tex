\documentclass[12pt]{article}
\usepackage[utf8]{inputenc}
\usepackage[dvips]{graphicx}
\usepackage{epsfig}
\usepackage{fancybox}
\usepackage{verbatim}
\usepackage{array}
\usepackage{latexsym}
\usepackage{alltt}
\usepackage{float}
\usepackage{amsmath}
\usepackage{hyperref}
\usepackage{listings}
\usepackage{color}
\usepackage[hmargin=3cm,vmargin=5.0cm]{geometry}
\topmargin=-1.8cm
\addtolength{\textheight}{6.5cm}
\addtolength{\textwidth}{2.0cm}
\setlength{\oddsidemargin}{0.0cm}
\setlength{\evensidemargin}{0.0cm}

\newcommand{\HRule}{\rule{\linewidth}{1mm}}
\newcommand{\kutu}[2]{\framebox[#1mm]{\rule[-2mm]{0mm}{#2mm}}}
\newcommand{\gap}{ \\[1mm] }

\newcommand{\Q}{\raisebox{1.7pt}{$\scriptstyle\bigcirc$}}

\lstset{
    %backgroundcolor=\color{lbcolor},
    tabsize=2,
    language=C++,
    basicstyle=\footnotesize,
    numberstyle=\footnotesize,
    aboveskip={0.0\baselineskip},
    belowskip={0.0\baselineskip},
    columns=fixed,
    showstringspaces=false,
    breaklines=true,
    prebreak=\raisebox{0ex}[0ex][0ex]{\ensuremath{\hookleftarrow}},
    %frame=single,
    showtabs=false,
    showspaces=false,
    showstringspaces=false,
    identifierstyle=\ttfamily,
    keywordstyle=\color[rgb]{0,0,1},
    commentstyle=\color[rgb]{0.133,0.545,0.133},
    stringstyle=\color[rgb]{0.627,0.126,0.941},
}


\begin{document}



\noindent
\HRule \\[3mm]
\small
\begin{tabular}[b]{lp{3.8cm}r}
{} Middle East Technical University &  &
{} Department of Computer Engineering \\
\end{tabular} \\
\begin{center}

                 \LARGE \textbf{CENG 223} \\[4mm]
                 \Large Discrete Computational Structures \\[4mm]
                \normalsize Fall '2020-2021 \\
                    \Large Homework 3 \\
                \normalsize Student Name and Surname: Berkin Kerim Konar  \\
                \normalsize Student Number:  2375343\\
\end{center}
\HRule


\section*{Question 1}

Answer:\\
   
    $2^{22} + 4^{44} + 6^{66} + 8^{80} + 10^{110} \mod 11 \equiv 2^{20}*2^2 + 4^{40}*4^4 + 6^{60}*6^6 + 8^{80} + 10^{110} \mod 11$\\
    
    By Fermat's Little Theorem we can conclude that:\\
    
    $2^{20} \mod 11 \equiv 4^{40} \mod 11 \equiv 6^{60} \mod 11 \equiv 8^{80} \mod 11 \equiv 10^{110} \mod 11 \equiv 1$\\
    
    So our equation reduces to:\\
    
    $2^2 + 4^4 + 6^6 + 1 + 1 \mod 11 \equiv 4 + 16^2 + 36^3 + 2 \mod 11 \equiv 4+5^2 + 3^3 + 2 \mod 11 \equiv$\\
   
    $4 + 25 + 27 + 2 \mod 11 \equiv 4 + 3 + 5 + 2 \mod 11 \equiv 14 \mod 11 \equiv 3$\\
    
    So the answer is 3.\\

\section*{Question 2}

Answer:\\

    $gcd(5n+3,7n+4) = gcd(5n+3,7n+4-5n-3) = gcd(5n+3,2n+1) = gcd(2n+1,5n+3-2*(2n+1)) =$\\
    
    $gcd(2n+1,n+1) = gcd(n+1,2n+1-n-1) = gcd(n+1,n) = gcd(n,n+1-n) = gcd(n,1) =$\\
    
    $gcd(1,n-n*1) = gcd(1,0) = 1$\\
    
    So the answer is 1.\\\\\\\\

\section*{Question 3}

Answer:\\

    $m^2 = n^2 + kx$ for m,n,k are integer and x is a prime.\\
    
    then  we see that:\\
    
    $ m^2-n^2 \mod x \equiv kx \mod x \equiv 0$
    
    If we write $m^2-n^2$ as $(m-n)*(m+n)$ we get.\\
    
    $(m-n)(m+n) \mod x \equiv 0$ since both m and n are integer let us write $m+n$ and, $m-n$ as a and, b which both a and, b are integers.\\
    
    $a*b \mod x \equiv 0$\\
    
    Then we see that since x is prime and both a and b are integers either a or b (or both) has to be in the form $tx$ where t is an integer.\\
    
    Since $a = (m+n)$ and, $b = (m-n)$ and, either a or b (or both) is in the form $tx$ where t is an integer we conclude that.\\
    
    $x|(m+n)$ or $x|(m-n)$.\\

\section*{Question 4}
    
Answer:\\
    
    First Step:\\
    
        for n = 1 we see that $1 = \frac{1*(3-1)}{2}$\\
        
    Inductive Step:\\
    
        For $n \in Z^+$, assume $1 + 4 + 7 + ... (3n-2) = \frac{n(3n-1)}{2}$\\
        
        We see that.\\
        
        $1 + 4 + 7 + .... + (3n-2) + (3n+1) = \frac{n(3n-1)}{2} + (3n + 1) = \frac{n(3n-1)}{2} + \frac{6n+2}{2}$\\
        
        $\frac{n(3n-1)}{2} + \frac{6n+2}{2} = \frac{3n^2 + 5n + 2}{2} = \frac{(n+1)(3n+2)}{2}$.\\
        
        We also see that if we write n+1 in the equation we assumed for $n \in Z^+$ we also get.\\
        
        $\frac{(n+1)(3n+2)}{2}$\\
        
        So if the assumption is true for $n$ then it is also true for $n+1$\\
    
    By induction the formula is true for all $n\in Z^+$

\end{document}

